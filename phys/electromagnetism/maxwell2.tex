\section{物質中の電磁場}

\subsection{電気双極子モーメント}
    大きさの等しい正と負の電荷が少しの距離を置いて存在している系を電気双極子と呼ぶ。電荷をそれぞれ$+q, -q$とする。負電荷の位置から正電荷の位置に引いたベクトルを$s$として、電気双極子モーメント$p$を
        \[p = qs\]
    と定義する。
    電気双極子が一様な電場$E$に置かれているとする。$s$と$E$のなす角が$\theta$であるときの電気双極子の位置エネルギーは$\theta = \pi / 2$のときを基準として
    \begin{align*}
        U &= -\left[q|E|\frac{s}{2}\cos\theta + (-q|E|) \cdot -\frac{s}{2}\cos\theta\right]\\
        &= -p \cdot E
    \end{align*}
    電気双極子の作る電位を求める。電気双極子の中心を原点として、正電荷と負電荷をそれぞれ$(-\frac{s}{2}, 0, 0), (+\frac{s}{2}, 0, 0)$の位置に置く。$r = (x, y, z)$における電位は
    \begin{align*}
        U(r) &= \frac{q}{4\pi\epsilon_0}\(\frac{1}{r_1} - \frac{1}{r_2}\)\\
        r_1 &= \sqrt{r^2 + (s/2)^2 - rs\cos\theta}\\
        r_2 &= \sqrt{r^2 + (s/2)^2 + rs\cos\theta}\\
    \end{align*}
    ここで$s \ll r$として近似すると
    \begin{align*}
        \frac{1}{r_1}
        &= \frac{1}{\sqrt{r^2 + (s/2)^2 - rs\cos\theta}}\\
        &= \frac{1}{r}\(1 + \(\frac{s}{2r}\)^2 - \frac{s}{r}\cos\theta\)^{-1/2}\\
        &\approx \frac{1}{r}\(1 - \frac{1}{2}\llr{\(\frac{s}{2r}\)^2 - \frac{s}{r}\cos\theta}\)\\
        &= \frac{1}{r}\(1 - \frac{s^2}{8r^2} + \frac{s}{2r}\cos\theta\)
    \end{align*}
    同様に
        \[\frac{1}{r_2} \approx \frac{1}{r}\(1 - \frac{s^2}{8r^2} - \frac{s}{2r}\cos\theta\)\]
    だから
    \begin{align*}
        U(r)
        &\approx \frac{q}{4\pi\epsilon_0}\frac{s}{r^2}\cos\theta\\
        &= \frac{p \cdot r}{4\pi\epsilon_0r^3}
    \end{align*}
    電場は
    \begin{align*}
        E_x
        &= -\pd{x}\(\frac{1}{4\pi\epsilon_0}\frac{p_x x + p_y y + p_z z}{(x^2+y^2+z^2)^{3/2}}\)\\
        &= -\frac{1}{4\pi\epsilon_0}\llr{\frac{p_x}{(x^2 + y^2 + z^2)^{3/2}} - \frac{3}{2}(p_x x + p_y y + p_z z)(x^2 + y^2 + z^2)^{-5/2}(2x)}\\
        &= -\frac{1}{4\pi\epsilon_0}\llr{\frac{p_x}{r^3} - \frac{3x(p \cdot r)}{r^5}}
    \end{align*}
    $y, z$も同様なので、結局
        \[E(r) = \frac{1}{4\pi\epsilon_0}\llr{\frac{p}{r^3} - \frac{3r(p\cdot r)}{r^5}}\]
    となる。

\subsection{磁気モーメント}
    モノポールが存在するとして、等しい大きさのNとSの磁荷を少しの距離を置いて配置したものを磁気双極子という。磁荷をそれぞれ$+q_m(N), -q_m(S)$とする。単位は$Wb$で、$F = q_m H$が成り立つ。S極からN極へ向かうベクトルを$s$として、磁気双極子モーメントと次のように定義する。
        \[m = q_m s\]
    電気双極子と同じように、エネルギーと磁場はそれぞれ、
    \begin{align*}
        U &= -m \cdot H\\
        H(r) &= -\frac{1}{4\pi\mu_0}\llr{\frac{m}{r^3} - \frac{3r(m\cdot r)}{r^5}}
    \end{align*}
    となる。

    ところが現代ではモノポールは存在しないとするのが主流なので、代わりに微小な円形電流を考える。この円形電流の作る磁場は十分遠方では磁気双極子が作るものと同じと見なせる。磁気モーメント$m'$を、
        \[U = -m' \cdot B\]
    が成り立つようにすると、$m' = m / \mu_0$とすれば良いことがわかる。すると磁束密度は、
        \[B(r) = -\frac{\mu_0}{4\pi}\llr{\frac{m'}{r^3} - \frac{3r(m'\cdot r)}{r^5}}\]
    となる。

\subsection{誘電体と電束密度}
    絶縁体(不導体)に静電場をかけると内部の電荷がその方向に偏る。これを分極という。絶縁体をそのように見たときこれを誘電体という。外部の電場によって誘電体が分極すると全体的に電荷が弱まる。そのとき単位面積あたりに通過した正電荷の量と方向を分極ベクトル$P$と呼ぶ。このとき積分形のガウスの法則は
        \[\int \epsilon_0 E \cdot dS = q - \int P \cdot dS\]
    となる。第二項は
        \[\int P \cdot dS = \int \div P dV\]
    と書き換えることができる。誘起された電荷密度$-\div P$を分極電荷という。電束密度を
        \[D = \epsilon_0 E + P\]
    と定義すると
    \begin{align*}
        \int (\epsilon_0 E + P) \cdot dS &= q\\
        \div D &= \rho\\
    \end{align*}
    となる。多くの物質では$P = \epsilon_0\chi E$という関係が近似的に成り立つ。$\chi$は電気感受率と呼ばれる。比較的弱い電場では電場に比例した分極を起こすが、大きい電場をかけると絶縁破壊が起きて電気が流れてしまう。$\epsilon = \epsilon_0(1 + \chi)$と置くと、$D = \epsilon E$と表せる。$\epsilon$を物質の誘電率という。常に$\chi > 0$なので$\epsilon > \epsilon_0$である。また$\epsilon / \epsilon_0$を比誘電率という。

\subsection{磁性体と磁場}
    永久磁石などの磁場の原因は電子が作る分子電流である。物質内部の分子電流は普段は別々の方向を向いているが、静磁場がかかると一つの向きに揃う。これを磁化という。物質を磁場に対する性質から見たとき、これを磁性体と呼ぶ。外部の磁場と同じ向きに磁場が誘起されるとき常磁性、逆向きのとき反磁性という。誘電体中で静磁場がかかると分子電流が向きを揃え、結果的に電流も変化する。このとき増加した磁気モーメントを磁化ベクトル$J$と呼ぶ。電流密度は
        \[\rot J = \mu_0i_m\]
    また、分極ベクトルの時間変化によっても電流が発生する。密度$\rho$の電荷が$u$だけ変位したとき、発生する電流は
    \begin{gather*}
        i_p = \rho\pd[u]{t} = \pd[P]{t}\\
        \rot B - \epsilon\mu_0\pd[E]{t} = \mu_0 (i + i_m)\\
    \end{gather*}
    $i_p$を分極電流と呼ぶ。磁場を
        \[H = \frac{1}{\mu_0}(B - J)\]
    と定義するとアンペールの法則は、
    \begin{align*}
        \frac{1}{\mu_0}\rot B - \epsilon_0 \pd[E]{t}
        &= i + i_m + i_p\\
        &= i + \frac{1}{\mu_0}\rot J + \pd[P]{t}\\
        \frac{1}{\mu_0}\rot(B - J) &- \pd[(\epsilon E + P)]{t} = i\\
        \rot H &- \pd[D]{t} = i\\
    \end{align*}
    となる。多くの物質では$J = \mu_0\chi_m H$という関係が近似的に成り立つ。$\chi_m$は磁化率と呼ばれる。磁場が一定以上強くなると分子電流の向きがすべて揃い、それ以上磁化ベクトルが大きくならないため、この比例関係は崩れる。$\mu = \mu_0(1 + \chi_m)$と置くと、$B = \mu H$と表せる。$\mu$を物質の透磁率という。磁性体の場合は$\mu$が正にも負にもなりうる。また$\mu / \mu_0$を比透磁率という。

\subsection{物質中のマクスウェル方程式}
    以上をまとめると物質中のマクスウェル方程式は次のようになる。
    \begin{align*}
        \div D &= \rho\\
        \div B &= 0\\
        \rot E &+ \pd[B]{t} = 0\\
        \rot H &- \pd[D]{t} =  i\\
    \end{align*}